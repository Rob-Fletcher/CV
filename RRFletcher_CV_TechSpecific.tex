
%----------------------------------------------------------------------------------------
%	PACKAGES AND OTHER DOCUMENT CONFIGURATIONS
%----------------------------------------------------------------------------------------

\documentclass[a4paper,10pt]{article} % Default font size and paper size

\usepackage{multirow}
\usepackage{longtable}

\usepackage[bottom=5mm,top=10mm]{geometry}
%\usepackage{fontspec} % For loading fonts
%\defaultfontfeatures{Mapping=tex-text}
%\setmainfont[SmallCapsFont = Fontin SmallCaps]{Fontin} % Main document font

\usepackage{xunicode,xltxtra,url,parskip} % Formatting packages

\usepackage[usenames,dvipsnames]{xcolor} % Required for specifying custom colors

%\usepackage[big]{layaureo} % Margin formatting of the A4 page, an alternative to layaureo can be \usepackage{fullpage}
% To reduce the height of the top margin uncomment: \addtolength{\voffset}{-1.3cm}

\usepackage{hyperref} % Required for adding links	and customizing them
\definecolor{linkcolour}{rgb}{0,0.2,0.6} % Link color
\hypersetup{colorlinks,breaklinks,urlcolor=linkcolour,linkcolor=linkcolour} % Set link colors throughout the document

\usepackage{titlesec} % Used to customize the \section command
\titleformat{\section}{\Large\scshape\raggedright}{}{0em}{}[\titlerule] % Text formatting of sections
\titlespacing{\section}{0pt}{3pt}{3pt} % Spacing around sections

\begin{document}

\pagestyle{empty} % Removes page numbering

\font\fb=''[cmr10]'' % Change the font of the \LaTeX command under the skills section

%----------------------------------------------------------------------------------------
%	NAME AND CONTACT INFORMATION
%----------------------------------------------------------------------------------------

\par{\centering{\Huge Rob Roy \textsc{Fletcher}}\bigskip\par} % Your name
\begin{centering}
\begin{tabular}{ l | c | r}
Philadelphia, PA   &  +1 951 858 2459   &   \href{mailto:robflet@sas.upenn.edu}{robflet@sas.upenn.edu}
\end{tabular} \par
\end{centering}

\vspace{5mm}
I am a current Ph.D. candidate in high energy physics working on the ATLAS experiment at the Large Hadron
Collider. My goal is to find a career in the technology industry that will allow me to use my software
and data analysis skills to help develop cutting edge software and solutions. I am looking for a collaborative,
team environment focused on innovation and problem solving.
\vspace{3mm}


%\section{Personal Data}

%\begin{tabular}{rl}
%\textsc{Place and Date of Birth:} & U.S.A  | 03 March 1984 \\
%\textsc{Address:} & 177 Gay St., Philadelphia, Pennsylvania, U.S.A. \\
%\textsc{Phone:} & +1 951 858 2459\\
%\textsc{email:} & \href{mailto:rob.fletcher@cern.ch}{rob.fletcher@cern.ch}
%\end{tabular}


%----------------------------------------------------------------------------------------
%	EDUCATION
%----------------------------------------------------------------------------------------

\section{Education}

\begin{tabular}{rl}
\textsc{Anticipated 2017} & Ph.D. Experimental High Energy Physics, \textbf{The University of Pennsylvania}, \\
                          & Philadelphia \\
&\small Advisor: Prof. I. Joseph \textsc{Kroll}\\
\vspace{1mm}

%---------------------------------------------------

\textsc{May} 2014 & Master of Science in \textsc{Physics}, \textbf{The University of Pennsylvania}, Philadelphia\\
&\small Advisor: Prof. I. Joseph \textsc{Kroll}\\
&\normalsize \textsc{Gpa}: 3.88/4.0\\
\vspace{1mm}

%------------------------------------------------

\textsc{June} 2012& Bachelor of Science in \textsc{}\textsc{Physics and Applied Mathematics} \\
&Cum Laude | \normalsize\textbf{The University of California}, Riverside\\
&\small Advisor: Prof. Gail \textsc{Hanson}\\
&\normalsize \textsc{Gpa}: 3.73/4.0 \\
&\\
%------------------------------------------------

\end{tabular}

%----------------------------------------------------------------------------------------
%	COMPUTER SKILLS
%----------------------------------------------------------------------------------------

\section{Computer Skills}

Experience writing code for large frameworks used in scientific computing. Participated
in management of code with version control systems (SVN, Git), issue tracking systems (JIRA, Git)
and documentation with Doxygen.

\begin{tabular}{rl}

Advanced Knowledge: & C++, Python, ROOT and PyROOT, Linux and Unix Systems \\
Intermediate Knowledge: & HTML, Javascript, CSS, SQL, NodeJS, LabView, AutoCAD, \LaTeX, \\
Basic Knowledge:        & PHP
\end{tabular}

%----------------------------------------------------------------------------------------
%	Hardware Skills
%----------------------------------------------------------------------------------------

\section{Hardware Skills}

Experience with Digital and Analogue circuits including several common microcontrollers (Arduino, Basic Series, Atmel chips) \\
Manual Machining Techniques (Mill, Lathe, etc.) TIG and MIG welding\\

%----------------------------------------------------------------------------------------

%----------------------------------------------------------------------------------------
%	WORK EXPERIENCE
%----------------------------------------------------------------------------------------

\section{Work Experience}

\begin{longtable}{r|p{11cm}}
\emph{Current}    & Ph.D. student at \textsc{University of Pennsylvania}, Philadelphia \\
\textsc{July 2012} & \emph{Student Researcher on ATLAS Experiment}\\
				   & \begin{itemize}
					 \footnotesize{
						\item Work in the Electron/Photon group focused on developing methods and software for a maximum likelihood based machine learning algorithm to identify electrons with high efficiency.
						\item Responsible for development of several combined performance software tools and their integration into the ATLAS Event Data Model.
						\item Currently work on a search for high-mass di-photon resonances targeted at identifying Randal-Sundrum type gravitons and extra dimensions.
						\item TA duties including grading and teaching undergraduate physics laboratories.
					 }
					 \end{itemize} \\
\multicolumn{2}{c}{} \\
%\end{tabular}

%------------------------------------------------

%\begin{tabular}{r|p{11cm}}
\textsc{March 2010-June 2012} & Undergraduate Researcher at \textsc{University of California}, Riverside \emph{}\\
				  & \begin{itemize}
					\footnotesize{
						\item Performed analysis on data collected by the Muon Ionization Cooling Experiment (MICE), both onsite at the Rutherford Appleton Laboratory (RAL) in Didcot, UK, and remotely from Riverside, CA
						\item Worked mainly on studying the contamination of neutral particles in the muon beam.
					    \item Developed software for the MAUS data analysis framework.  Worked on several parts of analysis code as well as a majority of the code that runs a set of three Time-of-Flight detectors.
						\item Worked on data collection shifts in the MICE control room at RAL.
					}
			        \end{itemize} \\
\multicolumn{2}{c}{} \\
%\end{tabular}

%------------------------------------------------

%\begin{tabular}{r|p{11cm}}
\textsc{Jan 2011-Mar 2011} & Suplemental Instructor, \textsc{Riverside Community College, STEM Center}, Riverside, CA \emph{}\\
		& \begin{itemize}
			\footnotesize{
				\item Supplemental Instructor for PHYS-4A – Classical Mechanics
				\item Responsible for 3 hours of lecture per week
   				\item Assisted with running the laboratory sessions
			}
          \end{itemize} \\
\end{longtable}

%----------------------------------------------------------------------------------------
%	Presentations
%----------------------------------------------------------------------------------------

\section{Presentations}

\begin{tabular}{rl}
\textsc{Aug.} 2014  & \textbf{Electron ID in Run 2}, US ATLAS meeting, University of Washington \\
                    &  Seattle, WA \\
\end{tabular}


%----------------------------------------------------------------------------------------
%	Selected Publications
%----------------------------------------------------------------------------------------

\section{Selected Publications}

\begin{tabular}{rl}
\textsc{Sept.} 2016 & \textbf{Search for resonances in diphoton events at $\sqrt{s}$=13 TeV} \\
									& \textbf{with the ATLAS detector} \\
									& The ATLAS collaboration, Aaboud, M., Aad, G. et al.  \\
									& J. High Energ. Phys. (2016) 2016: 1. doi:10.1007/JHEP09(2016)001, \\
									& arXiv:1606.03833v1 [hep-ex] \\
\textsc{Jun.} 2016 & \textbf{Electron efficiency measurements with the ATLAS detector} \\
									& \textbf{using the 2015 LHC proton-proton collision data} \\
									& ATLAS Collaboration, 51st Rencontres de Moriond on QCD \\
									& and High Energy Interactions, La Thuile, Italy, \\
									& https://cds.cern.ch/record/2157687 \\
\textsc{Oct.} 2015 & \textbf{Search for extra dimensions using diphoton events in} \\
									 & \textbf{13 TeV proton-proton collisions with the ATLAS detector:} \\
									 & \textbf{supporting document for the first analysis at 13 TeV} \\
									 & Leonardo Carminati et al, ATLAS Note \\
\textsc{May} 2012  & \textbf{The MICE Muon Beam on ISIS and the beam-line Instrumentation} \\
                   & \textbf{of the Muon Ionization Cooling Experiment}, M. Bogomilov et al., Journal of \\
				   & Instrumentation 2012 JINST 7 P05009, arXiv:1203.4089 \\
\textsc{Sept.} 2011 & \textbf{Proton Contamination Studies in the Muon Ionization Cooling Experiment} \\
					& S. Blot, Y.K. Kim, R.R. Fletcher, D. Kaplan, C. Rogers, Proceedings of the International \\
					& Particle Accelerator Conference 2011, San Sebastian, Spain \\
\textsc{March} 2011 & \textbf{Measurement of Neutral Particle Contamination in the MICE Muon Beam}\\
					& R. Fletcher, L. Coney, G. Hanson, Published in the Proceedes of the Particle \\
                    & Accelerator Conference 2011, New York, NY, arXiv:1105.0645\\

\end{tabular}

%----------------------------------------------------------------------------------------
%	Awards
%----------------------------------------------------------------------------------------

\section{Awards}

\begin{tabular}{rl}
\textsc{Sept.} 2016 & \textbf{3rd Place Overall and Taser/Axon sponsor prize for Best Public Safety} \\
					& \textbf{ and Video Processing App} for a user and object relative tracking transparent \\
					& heads up display (eyeHUD, http://devpost.com/software/eyehud). \\
\textsc{Jan.} 2016  & \textbf{1st Place PennAppsXIII} for implementation of an RF communication \\
 					& using motherboard RAM bus. \\
\textsc{Sept.} 2015 & \textbf{The Chairmans Teaching Award}, University of Pennsylvania \\
\textsc{June} 2012  & \textbf{Science Circle Award of Excelence}, University of California, Riverside \\
\textsc{May} 2012   & \textbf{Academic Excelence Award in Physics}, University of California, Riverside \\
\textsc{Sept.} 2011 & \textbf{Student Travel Grant}, International Particle Accelerator Conference \footnotesize(\$1,600)\normalsize\\
\textsc{May} 2010   & \textbf{Dean's Fellowship}, University of California, Riverside, College of Natural\\
				    & and Agricultural Sciences \footnotesize(\$2,000)\normalsize\\
\textsc{May} 2009   & \textbf{Summer Bridge to Research Grant}, University of California, Riverside, \\
			        & College of Natural and Agricultural Sciences \footnotesize(\$2,000)\normalsize\\

\end{tabular}


%----------------------------------------------------------------------------------------
%	LANGUAGES
%----------------------------------------------------------------------------------------

%\section{Languages}

%\begin{tabular}{rl}
%\textsc{English:} & Native\\

%\textsc{French:} & Basic \\
%\end{tabular}

%\newpage


%----------------------------------------------------------------------------------------

\end{document}
