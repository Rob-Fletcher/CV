
%----------------------------------------------------------------------------------------
%	PACKAGES AND OTHER DOCUMENT CONFIGURATIONS
%----------------------------------------------------------------------------------------

\documentclass[a4paper,10pt]{article} % Default font size and paper size

\usepackage{multirow}
\usepackage{longtable}

\usepackage[bottom=7mm,top=10mm]{geometry}
%\usepackage{fontspec} % For loading fonts
%\defaultfontfeatures{Mapping=tex-text}
%\setmainfont[SmallCapsFont = Fontin SmallCaps]{Fontin} % Main document font

\usepackage{xunicode,xltxtra,url,parskip} % Formatting packages

\usepackage[usenames,dvipsnames]{xcolor} % Required for specifying custom colors

%\usepackage[big]{layaureo} % Margin formatting of the A4 page, an alternative to layaureo can be \usepackage{fullpage}
% To reduce the height of the top margin uncomment: \addtolength{\voffset}{-1.3cm}

\usepackage{hyperref} % Required for adding links	and customizing them
\definecolor{linkcolour}{rgb}{0,0.2,0.6} % Link color
\hypersetup{colorlinks,breaklinks,urlcolor=linkcolour,linkcolor=linkcolour} % Set link colors throughout the document

\usepackage{titlesec} % Used to customize the \section command
\titleformat{\section}{\Large\scshape\raggedright}{}{0em}{}[\titlerule] % Text formatting of sections
\titlespacing{\section}{0pt}{2pt}{2pt} % Spacing around sections

\begin{document}

\pagestyle{empty} % Removes page numbering

\font\fb=''[cmr10]'' % Change the font of the \LaTeX command under the skills section

%----------------------------------------------------------------------------------------
%	NAME AND CONTACT INFORMATION
%----------------------------------------------------------------------------------------

\par{\centering{\Huge Rob Roy \textsc{Fletcher}}\bigskip\par} % Your name
\begin{centering}
\begin{tabular}{ l | c | c | r}
Redlands, CA   &  +1 951 858 2459   &   \href{mailto:rfletcher@seerai.space}{rfletcher@seerai.space} & \href{www.robroyfletcher.com}{www.robroyfletcher.com}
\end{tabular} \par
\end{centering}

\vspace{5mm}
I am an experienced data scientist with a background in experimental physics. I love to tackle new challenges and learn new 
skills. I have experience working in many different types of analysis and data science as well as leading teams of data 
scientists in delivering solutions.
\vspace{3mm}




%----------------------------------------------------------------------------------------
%	EDUCATION
%----------------------------------------------------------------------------------------

\section{Education}

\begin{tabular}{rl}
\textsc{Aug 2018} & Ph.D. Experimental High Energy Physics, \textbf{The University of Pennsylvania}, \\
                          & Philadelphia \\
&\small Advisor: Prof. I. Joseph \textsc{Kroll}\\
\vspace{1mm}

%---------------------------------------------------

\textsc{May} 2014 & Master of Science in \textsc{Physics}, \textbf{The University of Pennsylvania}, Philadelphia\\
&\small Advisor: Prof. I. Joseph \textsc{Kroll}\\
\vspace{1mm}

%------------------------------------------------

\textsc{June} 2012& Bachelor of Science in \textsc{}\textsc{Physics and Applied Mathematics} \\
&Cum Laude | \normalsize\textbf{The University of California}, Riverside\\
&\small Advisor: Prof. Gail \textsc{Hanson}\\
%------------------------------------------------

\end{tabular}
\vspace{0mm}

%----------------------------------------------------------------------------------------
%	COMPUTER SKILLS
%----------------------------------------------------------------------------------------

\section{Computer Skills}

Experience writing code for large frameworks used in scientific computing. Building ML/DL pipelines for distributed systems in production environments.

\begin{tabular}{rl}

Advanced Knowledge: & Go, Python, C++, Pytorch, Scikit-learn, Docker, Kubernetes, Linux  \\
Basic Knowledge: & HTML, CSS, Javascript, React\\  
\end{tabular}
\vspace{1mm}

%----------------------------------------------------------------------------------------
%	Hardware Skills
%----------------------------------------------------------------------------------------

\section{Hardware Skills}
Knowledge and experience with 3D printing, rapid prototyping and GCode. Using 3D modeling\\
software, I have designed, produced and tested components and assemblies.\\
Experience with digital and analogue circuits including several common microcontrollers (Arduino, Basic Series, Atmel chips) \\
\vspace{0mm}

%----------------------------------------------------------------------------------------

%----------------------------------------------------------------------------------------
%	WORK EXPERIENCE
%----------------------------------------------------------------------------------------

\section{Work Experience}

\begin{longtable}{r|p{11cm}}
%---------------------------------------------------------------------

\emph{Current}    & Chief Scientist \& Co-Founder - \textsc{SeerAI} \\
\textsc{Jan 2021} 
                & \begin{itemize}
					 \footnotesize{
						\item Lead data science and innovation team.
                        \item Design and build software and frameworks for ML and Geospatial computing.
					 }
					 \end{itemize} \\
\multicolumn{2}{c}{} \\
%\end{tabular}
%---------------------------------------------------------------
\textsc{Aug 2018-Jan 2021}  & Senior Data Scientist - \textsc{Esri}, Redlands, CA \\
				   & \begin{itemize}
					 \footnotesize{
						\item Technical leadership for the data science team in Professional Services.
                        \item Build and deliver high quality analyses and applications through consulting engagements
                        \item ML/AI subject matter expert and Tech lead for long term client projects.
					 }
					 \end{itemize} \\
\multicolumn{2}{c}{} \\

%---------------------------------------------------------------------

\textsc{July 2012-Aug 2018} & Research Assistant - \textsc{University of Pennsylvania}, Philadelphia  \\
          & \textsc{Cern}, Geneva, Switzerland \\
\textsc{} & \emph{Researcher on ATLAS Experiment}\\
				   & \begin{itemize}
					 \footnotesize{
						\item My dissertation work focused on a search for low-mass di-photon resonances using the two Higgs doublet model as a benchmark.
            \item Developed new background modeling technique based on Gaussian Process Regression and integrate it into a statistical model.
            \item Automated the validation of dataset transformations with web based reporting.
						\item I developed methods and software for a likelihood based classification and analysis.
						\item TA duties including grading and teaching undergraduate physics laboratories.
					 }
					 \end{itemize} \\
\multicolumn{2}{c}{} \\
%\end{tabular}

%------------------------------------------------
\pagebreak
%\begin{tabular}{r|p{11cm}}
\textsc{March 2010-June 2012} & Undergraduate Researcher - \textsc{University of California}, Riverside \emph{}\\
				  & \begin{itemize}
					\footnotesize{
						\item Performed analysis on data collected by the Muon Ionization Cooling Experiment (MICE), both onsite at the Rutherford Appleton Laboratory (RAL) in Didcot, UK, and remotely from Riverside, CA
						\item Studied the contamination of neutral particles in the muon beam.
					  \item Developed software for the MAUS data analysis framework.  Worked on several parts of analysis code as well as a majority of the code that runs a set of three Time-of-Flight detectors.
						\item Worked on data collection shifts in the MICE control room at RAL.
					}
			        \end{itemize} \\
\multicolumn{2}{c}{} \\
%\end{tabular}

%------------------------------------------------

\end{longtable}

%----------------------------------------------------------------------------------------
%	Selected Publications
%----------------------------------------------------------------------------------------

\section{Selected Publications}

\begin{tabular}{rl}
\textsc{July} 2017 & \textbf{Search for new phenomena in high-mass diphoton final states}\\
									 & \textbf{using 37 fb−1 of proton-proton collisions collected at s√=13 TeV with the ATLAS detector}\\
									 & The ATLAS collaboration  \\
									 & Phys. Lett. B 775 (2017) 105 \\
									 & arXiv:1707.04147 [hep-ex]  \\
\textsc{Jun.} 2016 & \textbf{Electron efficiency Measurements with the ATLAS Detector} \\
									& \textbf{using the 2015 LHC proton-proton collision data}, \\
									& ATLAS Collaboration, 51st Rencontres de Moriond on QCD \\
									& and High Energy Interactions, La Thuile, Italy, \\
									& https://cds.cern.ch/record/2157687 \\

\end{tabular}
\vspace{3mm}


\end{document}